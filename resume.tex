% resume.tex
% Bailey Forrest (baileycforrest@gmail.com)
%
% Based on this template:
% http://www.toofishes.net/blog/latex-resume-follow-up/
%

\documentclass[10pt, letterpaper]{article}
\usepackage[letterpaper, top=0.4in, bottom=0.4in, left=0.75in, right=0.75in]{geometry}
\usepackage{mdwlist}
\usepackage{textcomp} % Symbols (\textbullet)
\usepackage{tgpagella} % Alternative font
\usepackage{enumitem} % setlist
\pagestyle{empty}
\setlength{\tabcolsep}{0em} % No space between columns

\setlist{nolistsep} % remove spaces before and after itemize

\newcommand{\titleTopOffset}{-1.0em} % Space to add before section title
\newcommand{\titleBotOffset}{-0.4em} % space to add after section title

% indentsection style, used for sections that aren't already in lists
% that need indentation to the level of all text in the document
\newenvironment{indentsection}[1]%
{\begin{list}{}
    {\setlength{\leftmargin}{#1}}
    \item[]
}
{\end{list}}

% opposite of above; bump a section back toward the left margin
\newenvironment{unindentsection}[1]
{\begin{list}{}
    {\setlength{\leftmargin}{-0.5#1}}
    \item[]
}
{\end{list}}

% format two pieces of text, one left aligned and one right aligned
\newcommand{\headerrow}[2]
{\begin{tabular*}{\linewidth}{l@{\extracolsep{\fill}}r}
    #1 & #2 \\
\end{tabular*}}

\newcommand{\sectionheader}[1]
{\hrule
\vspace{\titleTopOffset}
\subsection*{#1}
\vspace{\titleBotOffset}}


% Content begins here
\begin{document}

\begin{center}
{\LARGE \textbf{Bailey Forrest}}

101 Butternut Court\ \ \textbullet \ \ Slippery Rock, PA 16057 \\
\texttt{bcforres@andrew.cmu.edu}\ \ \textbullet \ \ (724) 421 7974\ \ \textbullet \ \ \texttt{github.com/bcforres}\ \ \textbullet \ \ \texttt{linkedin.com/in/bcforres}
\end{center}

\sectionheader{Education}
\begin{itemize}
    \item
        \headerrow
            {\textbf{Carnegie Mellon University}}
            {\textbf{Pittsburgh, PA}}
        \\
        \headerrow
            {\emph{B.S. Electrical and Computer Engineering, Carnegie Institute of Technology}}
            {\emph{Class of 2015}}
        \begin{itemize*}
            \item GPA: 4.00 / 4.00
        \end{itemize*}


        \smallskip
        \textbf{Relevant Completed Coursework:}

        \begin{tabular*}{\linewidth}{p{0.5\linewidth}  p{0.5\linewidth}}
            15410 - Operating System Design and Impl. & 18447 - Computer Architecture \\
            15437 - Web Application Development & 18348 - Embedded Systems Engineering \\
            15462 - Computer Graphics & 18487 - Computer Security and Applied Crypto. \\
        \end{tabular*}


        \smallskip
        \textbf{Relevant Enrolled Coursework:}

        \begin{tabular*}{\linewidth}{p{0.5\linewidth}  p{0.5\linewidth}}
            15411 - Compiler Design & 18545 - Advanced Digital Design Project \\
            15440 - Distributed Systems & \\
        \end{tabular*}

\end{itemize}


\sectionheader{Skills}
\begin{indentsection}{\parindent}
    \hyphenpenalty=1000
    \begin{description*}
        \item[Languages/Scripts:]
            C, C++, Assembly (x86, ARM, HC12), Java, JavaScript, Python, SML,
            HTML, CSS, SQL

        \item[Hardware:]
            Verilog, System Verilog

        \item[Computers:]
            Version control (git, svn, hg, p4), Linux, Unix, \LaTeX,
            Computer Repair

    \end{description*}
\end{indentsection}


\sectionheader{Work Experience}
\begin{itemize}
    \item
        \headerrow
            {\textbf{Nvidia Corporation}}
            {\textbf{Santa Clara, CA}}
        \\
        \headerrow
            {\emph{Software Engineering Intern}}
            {\emph{Summer 2014}}
        \begin{itemize*}
            \item Worked on Unified Virtual Memory (UVM) windows driver. UVM
                allows CUDA applications to share the same pointer on both CPU
                and GPU accesses in order to avoid manual data migration

            \item Primary task was implementing the replayable fault handling
                infrastructure for the next generation GPU

            \item Performed other tasks such as writing tests and
                refactoring code

        \end{itemize*}

    \item
        \headerrow
            {\textbf{Spin Fusion}}
            {\textbf{Denver, CO}}
        \\
        \headerrow
            {\emph{Software Engineering Intern}}
            {\emph{Summer 2013}}
        \begin{itemize*}
            \item Worked on Spin Schedule web app, which provides schedule
                generation and management for hospitals

            \item Designed and led development of a REST API as a general
                purpose interface to the web application

            \item Worked other projects such as SMS notification services,
                single sign on, and front end features

        \end{itemize*}

    \item
        \headerrow
            {\textbf{CMU Computer Science Department}}
            {\textbf{Pittsburgh, PA}}
        \\
        \headerrow
            {\emph{Teaching Assistant for 15214 - Principles of Software Construction}}
            {\emph{Fall 2013}}
        \begin{itemize*}
            \item Teaching assistant for a course about software design
                principles and concurrency

            \item Led a weekly recitation on class materials

            \item Held other duties including creating course content, grading,
                and office hours
        \end{itemize*}

\end{itemize}


\sectionheader{Projects}
\begin{itemize}
    \item
        \headerrow
            {\textbf{UVM Fault Handler}}
            {\textbf{Summer 2014}}
        \\
        \headerrow
            {\emph{Nvidia Corporation}}
            {\emph{Tools Used: C}}
        \begin{itemize*}
            \item Implemented replayable fault handling infrastructure for UVM
                on Windows for the next generation GPU

            \item Worked extensively with the software chip simulator for the
                GPU, aided in support for Windows x64

            \item Wrote tests to ensure proper faulting behavior and handling
        \end{itemize*}

    \item
        \headerrow
            {\textbf{Pebbles Kernel and Utilities}}
            {\textbf{Spring 2014}}
        \\
        \headerrow
            {\emph{CMU 15410}}
            {\emph{Tools Used: C, x86 assembly}}
        \begin{itemize*}
            \item Designed and implemented a Unix-inspired OS kernel

            \item Implemented core kernel features such as virtual memory,
                resource management, thread and process management, program
                loader, and hardware drivers

            \item Implemented a multiboot compatible boot loader capable of
                loading pebbles and other OSs

            \item Implemented a userspace thread library including thread
                creation, management, and synchronization primitives (mutex,
                condition variable, semaphore, readers writers lock)
        \end{itemize*}

    \item
        \headerrow
            {\textbf{Pipelined ARM Processor}}
            {\textbf{Spring 2014}}
        \\
        \headerrow
            {\emph{CMU 18447}}
            {\emph{Tools Used: Verilog}}
        \begin{itemize*}
            \item Created a Verilog description for a pipelined
                microarchitecture implementing a subset of the ARM ISA

            \item Awarded bonus points for having the highest performance branch
                predictor in the class
        \end{itemize*}

    \item
        \headerrow
            {\textbf{Spin Schedule REST API}}
            {\textbf{Summer 2013}}
        \\
        \headerrow
            {\emph{Spin Fusion}}
            {\emph{Tools Used: Java, Jersey}}
        \begin{itemize*}
            \item Designed and led development of a REST API to provide a
                general purpose interface to the Spin Schedule web application's
                features through HTTP requests, easing development of internal
                and 3rd party apps

            \item Provided support for other interns while they used the API in
                their own projects
        \end{itemize*}

\end{itemize}

\end{document}
